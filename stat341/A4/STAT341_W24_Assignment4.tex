% Options for packages loaded elsewhere
\PassOptionsToPackage{unicode}{hyperref}
\PassOptionsToPackage{hyphens}{url}
\PassOptionsToPackage{dvipsnames,svgnames,x11names}{xcolor}
%
\documentclass[
]{article}
\usepackage{amsmath,amssymb}
\usepackage{iftex}
\ifPDFTeX
  \usepackage[T1]{fontenc}
  \usepackage[utf8]{inputenc}
  \usepackage{textcomp} % provide euro and other symbols
\else % if luatex or xetex
  \usepackage{unicode-math} % this also loads fontspec
  \defaultfontfeatures{Scale=MatchLowercase}
  \defaultfontfeatures[\rmfamily]{Ligatures=TeX,Scale=1}
\fi
\usepackage{lmodern}
\ifPDFTeX\else
  % xetex/luatex font selection
\fi
% Use upquote if available, for straight quotes in verbatim environments
\IfFileExists{upquote.sty}{\usepackage{upquote}}{}
\IfFileExists{microtype.sty}{% use microtype if available
  \usepackage[]{microtype}
  \UseMicrotypeSet[protrusion]{basicmath} % disable protrusion for tt fonts
}{}
\makeatletter
\@ifundefined{KOMAClassName}{% if non-KOMA class
  \IfFileExists{parskip.sty}{%
    \usepackage{parskip}
  }{% else
    \setlength{\parindent}{0pt}
    \setlength{\parskip}{6pt plus 2pt minus 1pt}}
}{% if KOMA class
  \KOMAoptions{parskip=half}}
\makeatother
\usepackage{xcolor}
\usepackage[margin=1in]{geometry}
\usepackage{color}
\usepackage{fancyvrb}
\newcommand{\VerbBar}{|}
\newcommand{\VERB}{\Verb[commandchars=\\\{\}]}
\DefineVerbatimEnvironment{Highlighting}{Verbatim}{commandchars=\\\{\}}
% Add ',fontsize=\small' for more characters per line
\usepackage{framed}
\definecolor{shadecolor}{RGB}{248,248,248}
\newenvironment{Shaded}{\begin{snugshade}}{\end{snugshade}}
\newcommand{\AlertTok}[1]{\textcolor[rgb]{0.94,0.16,0.16}{#1}}
\newcommand{\AnnotationTok}[1]{\textcolor[rgb]{0.56,0.35,0.01}{\textbf{\textit{#1}}}}
\newcommand{\AttributeTok}[1]{\textcolor[rgb]{0.13,0.29,0.53}{#1}}
\newcommand{\BaseNTok}[1]{\textcolor[rgb]{0.00,0.00,0.81}{#1}}
\newcommand{\BuiltInTok}[1]{#1}
\newcommand{\CharTok}[1]{\textcolor[rgb]{0.31,0.60,0.02}{#1}}
\newcommand{\CommentTok}[1]{\textcolor[rgb]{0.56,0.35,0.01}{\textit{#1}}}
\newcommand{\CommentVarTok}[1]{\textcolor[rgb]{0.56,0.35,0.01}{\textbf{\textit{#1}}}}
\newcommand{\ConstantTok}[1]{\textcolor[rgb]{0.56,0.35,0.01}{#1}}
\newcommand{\ControlFlowTok}[1]{\textcolor[rgb]{0.13,0.29,0.53}{\textbf{#1}}}
\newcommand{\DataTypeTok}[1]{\textcolor[rgb]{0.13,0.29,0.53}{#1}}
\newcommand{\DecValTok}[1]{\textcolor[rgb]{0.00,0.00,0.81}{#1}}
\newcommand{\DocumentationTok}[1]{\textcolor[rgb]{0.56,0.35,0.01}{\textbf{\textit{#1}}}}
\newcommand{\ErrorTok}[1]{\textcolor[rgb]{0.64,0.00,0.00}{\textbf{#1}}}
\newcommand{\ExtensionTok}[1]{#1}
\newcommand{\FloatTok}[1]{\textcolor[rgb]{0.00,0.00,0.81}{#1}}
\newcommand{\FunctionTok}[1]{\textcolor[rgb]{0.13,0.29,0.53}{\textbf{#1}}}
\newcommand{\ImportTok}[1]{#1}
\newcommand{\InformationTok}[1]{\textcolor[rgb]{0.56,0.35,0.01}{\textbf{\textit{#1}}}}
\newcommand{\KeywordTok}[1]{\textcolor[rgb]{0.13,0.29,0.53}{\textbf{#1}}}
\newcommand{\NormalTok}[1]{#1}
\newcommand{\OperatorTok}[1]{\textcolor[rgb]{0.81,0.36,0.00}{\textbf{#1}}}
\newcommand{\OtherTok}[1]{\textcolor[rgb]{0.56,0.35,0.01}{#1}}
\newcommand{\PreprocessorTok}[1]{\textcolor[rgb]{0.56,0.35,0.01}{\textit{#1}}}
\newcommand{\RegionMarkerTok}[1]{#1}
\newcommand{\SpecialCharTok}[1]{\textcolor[rgb]{0.81,0.36,0.00}{\textbf{#1}}}
\newcommand{\SpecialStringTok}[1]{\textcolor[rgb]{0.31,0.60,0.02}{#1}}
\newcommand{\StringTok}[1]{\textcolor[rgb]{0.31,0.60,0.02}{#1}}
\newcommand{\VariableTok}[1]{\textcolor[rgb]{0.00,0.00,0.00}{#1}}
\newcommand{\VerbatimStringTok}[1]{\textcolor[rgb]{0.31,0.60,0.02}{#1}}
\newcommand{\WarningTok}[1]{\textcolor[rgb]{0.56,0.35,0.01}{\textbf{\textit{#1}}}}
\usepackage{longtable,booktabs,array}
\usepackage{calc} % for calculating minipage widths
% Correct order of tables after \paragraph or \subparagraph
\usepackage{etoolbox}
\makeatletter
\patchcmd\longtable{\par}{\if@noskipsec\mbox{}\fi\par}{}{}
\makeatother
% Allow footnotes in longtable head/foot
\IfFileExists{footnotehyper.sty}{\usepackage{footnotehyper}}{\usepackage{footnote}}
\makesavenoteenv{longtable}
\usepackage{graphicx}
\makeatletter
\def\maxwidth{\ifdim\Gin@nat@width>\linewidth\linewidth\else\Gin@nat@width\fi}
\def\maxheight{\ifdim\Gin@nat@height>\textheight\textheight\else\Gin@nat@height\fi}
\makeatother
% Scale images if necessary, so that they will not overflow the page
% margins by default, and it is still possible to overwrite the defaults
% using explicit options in \includegraphics[width, height, ...]{}
\setkeys{Gin}{width=\maxwidth,height=\maxheight,keepaspectratio}
% Set default figure placement to htbp
\makeatletter
\def\fps@figure{htbp}
\makeatother
\setlength{\emergencystretch}{3em} % prevent overfull lines
\providecommand{\tightlist}{%
  \setlength{\itemsep}{0pt}\setlength{\parskip}{0pt}}
\setcounter{secnumdepth}{-\maxdimen} % remove section numbering
\ifLuaTeX
  \usepackage{selnolig}  % disable illegal ligatures
\fi
\IfFileExists{bookmark.sty}{\usepackage{bookmark}}{\usepackage{hyperref}}
\IfFileExists{xurl.sty}{\usepackage{xurl}}{} % add URL line breaks if available
\urlstyle{same}
\hypersetup{
  pdftitle={STAT 341: Assignment 4},
  colorlinks=true,
  linkcolor={Maroon},
  filecolor={Maroon},
  citecolor={Blue},
  urlcolor={blue},
  pdfcreator={LaTeX via pandoc}}

\title{STAT 341: Assignment 4}
\usepackage{etoolbox}
\makeatletter
\providecommand{\subtitle}[1]{% add subtitle to \maketitle
  \apptocmd{\@title}{\par {\large #1 \par}}{}{}
}
\makeatother
\subtitle{DUE: Friday, April 5, 2024 by 5:00pm EDT}
\author{}
\date{\vspace{-2.5em}}

\begin{document}
\maketitle

\(\;\) \(\;\) \(\;\) \(\;\)

\hypertarget{notes}{%
\subsection{NOTES}\label{notes}}

Your assignment must be submitted by the due date listed at the top of
this document, and it must be submitted electronically in .pdf format
via Crowdmark. This means that your responses for different question
parts should begin on separate pages of your .pdf file. Note that your
.pdf solution file must have been generated by R Markdown. Additionally:

\begin{itemize}
\item
  For mathematical questions: your solutions must be produced by LaTeX
  (from within R Markdown). Neither screenshots nor scanned/photographed
  handwritten solutions will be accepted -- these will receive zero
  points.
\item
  For computational questions: R code should always be included in your
  solution (via code chunks in R Markdown). If code is required and you
  provide none, you will receive zero points.

  \begin{itemize}
  \tightlist
  \item
    \textbf{Exception} any functions used in the notes or function
    glossary can loaded using \texttt{echo=FALSE} but any other code
    chunks should have \texttt{echo=TRUE}. e.g.~the code chuck loading
    \texttt{calculatePVmulti} can use \texttt{echo=FALSE} but chunks
    that call \texttt{calculatePVmulti} should have \texttt{echo=TRUE}.
  \end{itemize}
\item
  For interpretation questions: plain text (within R Markdown) is
  required. Text responses embedded as comments within code chunks will
  not be accepted.
\end{itemize}

Organization and comprehensibility is part of a full solution.
Consequently, points will be deducted for solutions that are not
organized and incomprehensible. Furthermore, if you submit your
assignment to Crowdmark, but you do so incorrectly in any way (e.g., you
upload your Question 2 solution in the Question 1 box), you will receive
a 5\% deduction (i.e., 5\% of the assignment's point total will be
deducted from your point total).

\newpage

\hypertarget{the-data}{%
\subsection{THE DATA}\label{the-data}}

\emph{Online controlled experiments} seek to use user-generated data to
test and improve internet-based products and services. Informally
referred to as A/B tests, these experiments are an indispensable tool
for major technology companies when it comes to maximizing revenue and
optimizing the user experience. Industry giants run hundreds of
experiments on millions of users every day, testing changes to websites,
services, and installed software; desktop and mobile devices; front- and
back-end product features; personalization and recommendations; and
monetization strategies. This type of experimentation is ubiquitous in
the tech industry. If you've been on the internet today, you've been a
subject in one of these experiments.

\begin{center}
\includegraphics[width=5in]{asos.png}
\end{center}

This assignment will have you, with randomization tests and bootstrapped
confidence intervals, analyze an A/B test from
\href{https://www.asos.com/}{ASOS.com}: an online fashion and cosmetics
retailer that targets young adults in North America, Europe, and
Oceania. ASOS recently published a
\href{https://osf.io/64jsb/}{repository of data from 78 experiments} run
on their website in 2019 and 2020. You will be working with data
associated with one of these experiments.

In particular, suppose interest lies in evaluating whether adding a
``free delivery'' banner to the homepage materially impacts users'
shopping behaviour. To assess this, an experiment is run in which some
users to the ASOS homepage see such a banner (these users are in the
\emph{treatment} group), while other users see the original website with
no mention of free shipping (these users are in the \emph{control}
group). Whether a user is in the treatment or control group is random,
so any difference in purchasing behaviour between these groups can be
attributed to the presence/absence of the ``free delivery'' banner.
Purchase behaviour may be quantified in many ways (e.g., by recording
whether a purchase was made, the number of items purchased, or the total
order value).

In this experiment, the focus is on determining whether the ``free
delivery'' banner is associated with an increase in total order value.
The experiment ran for 19 days and engaged over 2 million users to
ASOS.com. However, to keep calculations manageable, you will work with
observations on just \(n=5,536\) users. The table below describes the
variates recorded for each of these users. This data is available in the
file \texttt{asos.csv}.

\begin{longtable}[]{@{}
  >{\raggedright\arraybackslash}p{(\columnwidth - 2\tabcolsep) * \real{0.0904}}
  >{\raggedright\arraybackslash}p{(\columnwidth - 2\tabcolsep) * \real{0.9096}}@{}}
\toprule\noalign{}
\begin{minipage}[b]{\linewidth}\raggedright
Variate
\end{minipage} & \begin{minipage}[b]{\linewidth}\raggedright
Description
\end{minipage} \\
\midrule\noalign{}
\endhead
\bottomrule\noalign{}
\endlastfoot
\texttt{Day} & This variate takes on the values \texttt{1:19} indicating
which day of the experiment the user was observed. \\
\texttt{Order.Value} & This is a numeric variate indicating the total
amount spent (in £) by the user. \\
\texttt{Version} & This is a categorical variate taking on the values
\texttt{"T"} or \texttt{"C"} respectively indicating whether the user
was in the \emph{treatment} or \emph{control} group. \\
\end{longtable}

\newpage

\hypertarget{question-1-randomization-tests-21-points}{%
\subsection{QUESTION 1: Randomization Tests {[}21
points{]}}\label{question-1-randomization-tests-21-points}}

\begin{enumerate}
\def\labelenumi{(\alph{enumi})}
\tightlist
\item
  {[}2 points{]} Read in the data and calculate a \texttt{summary()} of
  \texttt{Order.Value} for the treatment and control groups separately.
\end{enumerate}

\begin{Shaded}
\begin{Highlighting}[]
\NormalTok{data }\OtherTok{\textless{}{-}} \FunctionTok{read.csv}\NormalTok{(}\StringTok{"asos.csv"}\NormalTok{)}
\FunctionTok{summary}\NormalTok{(data[data}\SpecialCharTok{$}\NormalTok{Version }\SpecialCharTok{==} \StringTok{"T"}\NormalTok{,])}
\end{Highlighting}
\end{Shaded}

\begin{verbatim}
##       Day          Order.Value        Version         
##  Min.   : 1.000   Min.   :   0.00   Length:2768       
##  1st Qu.: 4.000   1st Qu.:   0.00   Class :character  
##  Median : 7.000   Median :   0.00   Mode  :character  
##  Mean   : 8.124   Mean   :  27.07                     
##  3rd Qu.:12.000   3rd Qu.:  14.00                     
##  Max.   :19.000   Max.   :1506.00
\end{verbatim}

\begin{Shaded}
\begin{Highlighting}[]
\FunctionTok{summary}\NormalTok{(data[data}\SpecialCharTok{$}\NormalTok{Version }\SpecialCharTok{==} \StringTok{"C"}\NormalTok{,])}
\end{Highlighting}
\end{Shaded}

\begin{verbatim}
##       Day          Order.Value        Version         
##  Min.   : 1.000   Min.   :   0.00   Length:2768       
##  1st Qu.: 4.000   1st Qu.:   0.00   Class :character  
##  Median : 7.000   Median :   0.00   Mode  :character  
##  Mean   : 8.127   Mean   :  27.37                     
##  3rd Qu.:12.000   3rd Qu.:  14.00                     
##  Max.   :19.000   Max.   :1470.00
\end{verbatim}

\begin{enumerate}
\def\labelenumi{(\alph{enumi})}
\setcounter{enumi}{1}
\tightlist
\item
  {[}4 points{]} Construct a QQ (i.e., quantile-quantile) plot comparing
  the quantiles of \texttt{Order.Value} in the treatment and control
  groups. Specifically, calculate quantiles of \texttt{Order.Value} for
  \texttt{p\ =\ seq(from=0,\ to=1,\ by=0.01)} for both groups, and
  create a scatter plot of these pairs with control quantiles on the
  \(x\)-axis. Add the line of equality (\(y=x\)) to the plot. Be sure to
  include an informative title and axis labels. Note that you \emph{may
  not use} the \texttt{qqplot()} function, but you can use it to check
  your answer.
\end{enumerate}

\begin{Shaded}
\begin{Highlighting}[]
\NormalTok{p }\OtherTok{=} \FunctionTok{seq}\NormalTok{(}\AttributeTok{from =} \DecValTok{0}\NormalTok{, }\AttributeTok{to =} \DecValTok{1}\NormalTok{, }\AttributeTok{by =} \FloatTok{0.01}\NormalTok{)}
\NormalTok{x }\OtherTok{=} \FunctionTok{quantile}\NormalTok{(data[data}\SpecialCharTok{$}\NormalTok{Version }\SpecialCharTok{==} \StringTok{"C"}\NormalTok{,]}\SpecialCharTok{$}\NormalTok{Order.Value, p)}
\NormalTok{y }\OtherTok{=} \FunctionTok{quantile}\NormalTok{(data[data}\SpecialCharTok{$}\NormalTok{Version }\SpecialCharTok{==} \StringTok{"T"}\NormalTok{,]}\SpecialCharTok{$}\NormalTok{Order.Value, p)}

\FunctionTok{plot}\NormalTok{(x,y, }\AttributeTok{xlab =} \StringTok{\textquotesingle{}Control Quantiles\textquotesingle{}}\NormalTok{, }\AttributeTok{ylab =} \StringTok{\textquotesingle{}Order Value\textquotesingle{}}\NormalTok{, }\AttributeTok{main=}\StringTok{\textquotesingle{}Control vs Order Quantiles\textquotesingle{}}\NormalTok{)}
\FunctionTok{abline}\NormalTok{(}\AttributeTok{a =} \DecValTok{0}\NormalTok{, }\AttributeTok{b =} \DecValTok{1}\NormalTok{)}
\end{Highlighting}
\end{Shaded}

\includegraphics{STAT341_W24_Assignment4_files/figure-latex/unnamed-chunk-3-1.pdf}

\begin{enumerate}
\def\labelenumi{(\alph{enumi})}
\setcounter{enumi}{2}
\item
  {[}2 points{]} By addressing both the summaries in (a) and the plot in
  (b), comment on the following:

  \begin{itemize}
  \tightlist
  \item
    What appears to be a typical \texttt{Order.Value} in each group?
  \item
    Does \texttt{Order.Value} appear to depend on whether a ``free
    delivery'' banner is shown?
  \end{itemize}
\end{enumerate}

\begin{itemize}
\tightlist
\item
  A typical Order.Value in each group is 0-500.
\item
  No
\end{itemize}

\begin{enumerate}
\def\labelenumi{(\alph{enumi})}
\setcounter{enumi}{3}
\tightlist
\item
  {[}1 point{]} Define \(\mathcal{S}_{T}\) and \(\mathcal{S}_{C}\) as
  the \texttt{Order.Value} observations for users in the treatment and
  control groups, respectively\footnote{Note that we use sample notation
    (i.e., \(\mathcal{S}_{T}\) and \(\mathcal{S}_{C}\)) here instead of
    population notation (i.e., \(\mathcal{P}_{T}\) and
    \(\mathcal{P}_{C}\)) to acknowledge that the users in the experiment
    are a \emph{subset} of all of the ASOS.com users.}. State the null
  hypothesis \(H_0\) that is being tested when comparing these two
  sub-populations with a randomization test.
\end{enumerate}

\(H_0\) is defined as there is no difference between the treatment and
control groups. Thus, \(H_0: \mathcal{S}_{T} =\mathcal{S}_{C}\)

\(\;\)

\begin{enumerate}
\def\labelenumi{(\alph{enumi})}
\setcounter{enumi}{4}
\item
  {[}5 points{]} In this question, you will test the hypothesis in (d)
  using the discrepancy measure
  \[D(\mathcal{S}_{T},\mathcal{S}_{C}) = |\overline{y}_T - \overline{y}_C|\]

  \begin{enumerate}
  \def\labelenumii{\roman{enumii}.}
  \tightlist
  \item
    {[}1 point{]} Calculate the observed discrepancy.
  \end{enumerate}

\begin{Shaded}
\begin{Highlighting}[]
\NormalTok{mean\_T }\OtherTok{\textless{}{-}} \FunctionTok{mean}\NormalTok{(data[data}\SpecialCharTok{$}\NormalTok{Version }\SpecialCharTok{==} \StringTok{"T"}\NormalTok{,]}\SpecialCharTok{$}\NormalTok{Order.Value)}
\NormalTok{mean\_C }\OtherTok{\textless{}{-}} \FunctionTok{mean}\NormalTok{(data[data}\SpecialCharTok{$}\NormalTok{Version }\SpecialCharTok{==} \StringTok{"C"}\NormalTok{,]}\SpecialCharTok{$}\NormalTok{Order.Value)}
\NormalTok{D }\OtherTok{\textless{}{-}} \FunctionTok{abs}\NormalTok{(mean\_T }\SpecialCharTok{{-}}\NormalTok{ mean\_C)}
\NormalTok{D}
\end{Highlighting}
\end{Shaded}

\begin{verbatim}
## [1] 0.296604
\end{verbatim}

  \begin{enumerate}
  \def\labelenumii{\roman{enumii}.}
  \setcounter{enumii}{1}
  \tightlist
  \item
    {[}2 points{]} Randomly mix the populations \(M=1000\) times and
    construct a histogram of the 1000
    \(D(\mathcal{S}_{T}^\star,\mathcal{S}_{C}^\star)\) values. Indicate,
    with a vertical line, the observed discrepancy calculated in i. Note
    that you may use the \texttt{mixRandomly()} function from class. For
    your convenience, the function is included in the Appendix at the
    end of the assignment.
  \end{enumerate}
\end{enumerate}

\begin{Shaded}
\begin{Highlighting}[]
\NormalTok{  mixRandomly }\OtherTok{\textless{}{-}} \ControlFlowTok{function}\NormalTok{(p) \{}
\NormalTok{  p1 }\OtherTok{\textless{}{-}}\NormalTok{ p}\SpecialCharTok{$}\NormalTok{pop1}
\NormalTok{  n1 }\OtherTok{\textless{}{-}} \FunctionTok{nrow}\NormalTok{(p1)}
\NormalTok{  p2 }\OtherTok{\textless{}{-}}\NormalTok{ p}\SpecialCharTok{$}\NormalTok{pop2}
\NormalTok{  n2 }\OtherTok{\textless{}{-}} \FunctionTok{length}\NormalTok{(p2)}
\NormalTok{  mix }\OtherTok{\textless{}{-}} \FunctionTok{rbind}\NormalTok{(p1, p2)}
\NormalTok{  select }\OtherTok{\textless{}{-}} \FunctionTok{sample}\NormalTok{(}\DecValTok{1}\SpecialCharTok{:}\NormalTok{(n1 }\SpecialCharTok{+}\NormalTok{ n2), n1, }\AttributeTok{replace =} \ConstantTok{FALSE}\NormalTok{) }
\NormalTok{  newp1 }\OtherTok{\textless{}{-}}\NormalTok{ mix[select,]}
\NormalTok{  newp2 }\OtherTok{\textless{}{-}}\NormalTok{ mix[}\SpecialCharTok{{-}}\NormalTok{select,]}
  \FunctionTok{list}\NormalTok{(}\AttributeTok{pop1 =}\NormalTok{ newp1, }\AttributeTok{pop2 =}\NormalTok{ newp2)}
\NormalTok{\}}

\NormalTok{x }\OtherTok{\textless{}{-}} \FunctionTok{c}\NormalTok{()}
\ControlFlowTok{for}\NormalTok{ (i }\ControlFlowTok{in} \DecValTok{1}\SpecialCharTok{:}\DecValTok{1000}\NormalTok{) \{}
\NormalTok{    p }\OtherTok{\textless{}{-}} \FunctionTok{mixRandomly}\NormalTok{(}\FunctionTok{list}\NormalTok{(}\AttributeTok{pop1 =}\NormalTok{ data[data}\SpecialCharTok{$}\NormalTok{Version }\SpecialCharTok{==} \StringTok{"T"}\NormalTok{,],}
    \AttributeTok{pop2=}\NormalTok{data[data}\SpecialCharTok{$}\NormalTok{Version }\SpecialCharTok{==} \StringTok{"C"}\NormalTok{,]))}
\NormalTok{    dis }\OtherTok{\textless{}{-}} \FunctionTok{abs}\NormalTok{(}\FunctionTok{mean}\NormalTok{(p}\SpecialCharTok{$}\NormalTok{pop1}\SpecialCharTok{$}\NormalTok{Order.Value) }\SpecialCharTok{{-}} \FunctionTok{mean}\NormalTok{(p}\SpecialCharTok{$}\NormalTok{pop2}\SpecialCharTok{$}\NormalTok{Order.Value))}
\NormalTok{    x }\OtherTok{\textless{}{-}} \FunctionTok{c}\NormalTok{(x, dis)}
\NormalTok{\}}
\FunctionTok{hist}\NormalTok{(x, }\AttributeTok{xlab=}\StringTok{"Idx"}\NormalTok{, }\AttributeTok{ylab=}\StringTok{"Discrepancy Measures"}\NormalTok{, }\AttributeTok{main=}\StringTok{"Histogram of Discrepancy Measures"}\NormalTok{,}\AttributeTok{prob=}\ConstantTok{TRUE}\NormalTok{)}
\FunctionTok{abline}\NormalTok{(}\AttributeTok{v=}\NormalTok{D,}\AttributeTok{col=}\StringTok{"RED"}\NormalTok{,}\AttributeTok{lwd=}\DecValTok{2}\NormalTok{)}
\end{Highlighting}
\end{Shaded}

\includegraphics{STAT341_W24_Assignment4_files/figure-latex/unnamed-chunk-5-1.pdf}

\begin{verbatim}
iii. [1 point] Calculate the $p\text{-value}$ associated with this test.
\end{verbatim}

\begin{Shaded}
\begin{Highlighting}[]
\NormalTok{p }\OtherTok{\textless{}{-}} \FunctionTok{mean}\NormalTok{(x }\SpecialCharTok{\textgreater{}=}\NormalTok{ D)}
\NormalTok{p}
\end{Highlighting}
\end{Shaded}

\begin{verbatim}
## [1] 0.465
\end{verbatim}

\begin{verbatim}
iv. [1 point] Based on the $p\text{-value}$ calculated in iii., what do you conclude about the comparability of these two sub-populations? In other words, summarize your findings and draw a conclusion about the null hypothesis from part (d). By referring to the summaries calculated in part (a), explain why this conclusion is, or is not, surprising.
\end{verbatim}

There is not enough evidence to reject the null hypothesis. This is not
very surprising since the summaries in part a) are quite similar.

\begin{enumerate}
\def\labelenumi{(\alph{enumi})}
\setcounter{enumi}{5}
\item
  {[}5 points{]} The comparison in (e) was based just on means. For a
  more holistic comparison of the \texttt{Order.Value} distributions, we
  could compare several quantiles, such as \(Q(0)\), \(Q(0.25)\),
  \(Q(0.5)\), \(Q(0.75)\), \(Q(1)\). However, increasing the number of
  comparisons increases the magnitude of the \emph{multiple testing}
  problem and so we must be careful to account for this. In this
  question, you will test the hypothesis in (d) using the discrepancy
  measure \[D(\mathcal{S}_{T},\mathcal{S}_{C}) = |Q_T(p) - Q_C(p)|\] for
  \(p\in\{0,0.25,0.5,0.75,1\}\), and you will account for the multiple
  testing problem by using the \texttt{calculatePVmulti()} function from
  class. For your convenience, the function is included in the Appendix
  at the end of the assignment.

  Also defined below is \texttt{getQCompFn}, a factory function of a
  single input \texttt{p} that outputs a function which calculates the
  absolute difference in \texttt{p} quantiles of \texttt{Order.Value}
  between the treatment and control groups. This will be useful in part
  i.~below.

\begin{Shaded}
\begin{Highlighting}[]
\NormalTok{getQCompFn }\OtherTok{\textless{}{-}} \ControlFlowTok{function}\NormalTok{(p)\{}
 \ControlFlowTok{function}\NormalTok{(pop) \{}
  \FunctionTok{as.numeric}\NormalTok{(}\FunctionTok{abs}\NormalTok{(}\FunctionTok{quantile}\NormalTok{(pop[[}\DecValTok{2}\NormalTok{]]}\SpecialCharTok{$}\NormalTok{Order.Value, p) }\SpecialCharTok{{-}} \FunctionTok{quantile}\NormalTok{(pop[[}\DecValTok{1}\NormalTok{]]}\SpecialCharTok{$}\NormalTok{Order.Value, p)))}
\NormalTok{ \}}
\NormalTok{\}}
\end{Highlighting}
\end{Shaded}

  \begin{enumerate}
  \def\labelenumii{\roman{enumii}.}
  \item
    {[}1 point{]} Use the factory function \texttt{getQCompFn} to create
    5 functions \texttt{absDiffQ0}, \texttt{absDiffQ25},
    \texttt{absDiffQ50}, \texttt{absDiffQ75}, and \texttt{absDiffQ100}
    which each take as input a 2-element list that respectively return
    the absolute difference in quantiles for treatment vs.~control for
    \(p=0,0.25,0.5,0.75,1\). Then use each these functions to calculate
    \(D(\mathcal{S}_{T},\mathcal{S}_{C}) = |Q_T(p) - Q_C(p)|\) for each
    \(p\).
  \item
    {[}2 points{]} Estimate \(p\text{-value}^{\star}\), the
    \(p\text{-value}\) associated with a test of \(H_0\) from part (d),
    that simultaneously accounts for the five discrepancy measures
    defined in ii. Use the \texttt{calculatePVmulti()} together with the
    discrepancy functions you defined in i., and use
    \texttt{M\_inner\ =\ M\_outer\ =\ 100}.
  \item
    {[}2 points{]} Based on \(p\text{-value}^{\star}\) calculated in
    ii., what do you conclude about the comparability of these two
    sub-populations? In other words, summarize your findings and draw a
    conclusion about the null hypothesis from part (d). By referring to
    the QQ-plot constructed in part (b), explain why this conclusion is,
    or is not, surprising.
  \end{enumerate}
\end{enumerate}

\(\;\)

\begin{enumerate}
\def\labelenumi{(\alph{enumi})}
\setcounter{enumi}{6}
\tightlist
\item
  {[}2 points{]} Explain in your own words what the \emph{multiple
  testing} problem is, and briefly explain why the approach taken in
  part (f) is to be preferred to considering five separate tests based
  on \(|Q_T(0) - Q_C(0)|\), \(|Q_T(0.25) - Q_C(0.25)|\),
  \(|Q_T(0.5) - Q_C(0.5)|\), \(|Q_T(0.75) - Q_C(0.75)|\), and
  \(|Q_T(1) - Q_C(1)|\) individually.
\end{enumerate}

\newpage

\hypertarget{question-2-bootstrap-confidence-intervals-25-points}{%
\subsection{QUESTION 2: Bootstrap Confidence Intervals {[}25
points{]}}\label{question-2-bootstrap-confidence-intervals-25-points}}

In Question 1, you conducted hypothesis tests to determine whether there
was a significant difference in order values when users are vs.~are not
shown a ``free delivery'' banner. In this question, you will turn your
focus to (point and interval) estimation of the \emph{treatment effect},
a quantification of the impact on \texttt{Order.Value} of the treatment
relative to the control. Treatment impact can be quantified in a variety
of ways; here you will focus on the \emph{average treatment effect}
\[\text{ATE} = \overline{y}_T-\overline{y}_C\] and also the percent
treatment effect, referred to in A/B testing settings as \emph{lift}:
\[\text{lift} = \frac{\overline{y}_T-\overline{y}_C}{\overline{y}_C}\]
where
\[\overline{y}_T = \frac{1}{n_T}\sum_{u\in\mathcal{S}_T}y_u ~~~~ \text{and} ~~~~ \overline{y}_C = \frac{1}{n_C}\sum_{u\in\mathcal{S}_C}y_u\]
are the average order values in the treatment and control groups,
respectively.

\begin{enumerate}
\def\labelenumi{(\alph{enumi})}
\item
  {[}4 points{]} Consider the following simple linear regression model
  \[y_u = \alpha + \beta~x_u+r_u, ~~~~ u\in\mathcal{S}=\mathcal{S}_T\cup\mathcal{S}_C\]
  where \(y_u\) represents the \texttt{Order.Value} of user \(u\) and
  \(x_u\) is a treatment assignment indicator defined as follows:
  \[x_u =
    \begin{cases}
  1 & \text{if user}~u~\text{is in the treatment group} \\
  0 & \text{if user}~u~\text{is in the control group}
    \end{cases}.\] Earlier in the course, it was established that the
  least squares estimates of \(\alpha\) and \(\beta\) are
  \[\hat\alpha=\overline{y}-\hat\beta~\overline{x} ~~~~ \text{and} ~~~~ \hat\beta=\frac{\sum_{u\in\mathcal{S}}(x_u-\overline{x})(y_u-\overline{y})}{\sum_{u\in\mathcal{S}}(x_u-\overline{x})^2}\]
  where \(\overline{x} = \sum_{u\in\mathcal{S}}x_u/n\),
  \(\overline{y} = \sum_{u\in\mathcal{S}}y_u/n\), and \(n= n_T + n_C\).

  \(\;\)

  For \(x_u\) and \(y_u\) as defined above, show that
  \[\hat\alpha = \overline{y}_C~~~~ \text{and} ~~~~\hat\beta=\overline{y}_T-\overline{y}_C\]
  and hence that \(\text{ATE}=\hat\beta\) and
  \(\text{lift}=\hat\beta/\hat\alpha\).

  \textbf{Hints:}

  \begin{itemize}
  \tightlist
  \item
    \(\sum_{u\in\mathcal{S}}x_u = n_T\)
  \item
    \(\sum_{u\in\mathcal{S}}y_ux_u = \sum_{u\in\mathcal{S}_T}y_u = n_T\overline{y}_T\)
  \item
    \(x_u^2=x_u ~~\forall~ u\)
  \end{itemize}
\end{enumerate}

\(\;\)

\[
\hat\alpha
\]

\begin{enumerate}
\def\labelenumi{(\alph{enumi})}
\setcounter{enumi}{1}
\tightlist
\item
  {[}2 points{]} Using the \texttt{lm()} function, fit the linear
  regression model above and calculate the \(\text{ATE}\) and
  \(\text{lift}\) for this A/B test.
\end{enumerate}

\begin{Shaded}
\begin{Highlighting}[]
\NormalTok{data}\SpecialCharTok{$}\NormalTok{Version }\OtherTok{\textless{}{-}} \FunctionTok{ifelse}\NormalTok{(data}\SpecialCharTok{$}\NormalTok{Version }\SpecialCharTok{==} \StringTok{"T"}\NormalTok{, }\DecValTok{1}\NormalTok{, }\DecValTok{0}\NormalTok{)}
\NormalTok{fit }\OtherTok{\textless{}{-}} \FunctionTok{lm}\NormalTok{(data}\SpecialCharTok{$}\NormalTok{Order.Value }\SpecialCharTok{\textasciitilde{}}\NormalTok{ data}\SpecialCharTok{$}\NormalTok{Version)}
\NormalTok{coeffs }\OtherTok{\textless{}{-}} \FunctionTok{coefficients}\NormalTok{(fit)}
\NormalTok{alpha }\OtherTok{\textless{}{-}}\NormalTok{ coeffs[[}\DecValTok{1}\NormalTok{]]; beta }\OtherTok{\textless{}{-}}\NormalTok{ coeffs[[}\DecValTok{2}\NormalTok{]]}
\NormalTok{ATE }\OtherTok{\textless{}{-}}\NormalTok{ beta}
\NormalTok{lift }\OtherTok{\textless{}{-}}\NormalTok{ beta}\SpecialCharTok{/}\NormalTok{alpha}
\FunctionTok{print}\NormalTok{(}\FunctionTok{paste0}\NormalTok{(}\StringTok{"ATE: "}\NormalTok{, ATE))}
\end{Highlighting}
\end{Shaded}

\begin{verbatim}
## [1] "ATE: -0.296604046242904"
\end{verbatim}

\begin{Shaded}
\begin{Highlighting}[]
\FunctionTok{print}\NormalTok{(}\FunctionTok{paste0}\NormalTok{(}\StringTok{"lift: "}\NormalTok{, lift))}
\end{Highlighting}
\end{Shaded}

\begin{verbatim}
## [1] "lift: -0.0108377115399895"
\end{verbatim}

\begin{enumerate}
\def\labelenumi{(\alph{enumi})}
\setcounter{enumi}{2}
\tightlist
\item
  {[}2 points{]} By resampling \(\mathcal{S}\) with replacement,
  construct \(B=1000\) bootstrap samples
  \(\mathcal{S}_1^\star,\mathcal{S}_2^\star,\ldots,\mathcal{S}_{1000}^\star\).
  Print out your code, not the 1000 samples. Note that the units in
  \(\mathcal{S}\) should be regarded as the \((x_u,y_u)\) pairs. Thus
  the pairs themselves should be resampled with replacement.
\end{enumerate}

\begin{Shaded}
\begin{Highlighting}[]
\NormalTok{B }\OtherTok{\textless{}{-}} \DecValTok{1000}
\NormalTok{samples }\OtherTok{\textless{}{-}} \FunctionTok{list}\NormalTok{()}
\ControlFlowTok{for}\NormalTok{ (b }\ControlFlowTok{in} \DecValTok{1}\SpecialCharTok{:}\NormalTok{B) \{}
\NormalTok{  samples[[b]] }\OtherTok{\textless{}{-}}\NormalTok{ data[}\FunctionTok{sample}\NormalTok{(}\FunctionTok{nrow}\NormalTok{(data), }\FunctionTok{nrow}\NormalTok{(data), }\AttributeTok{replace=}\ConstantTok{TRUE}\NormalTok{), }\FunctionTok{c}\NormalTok{(}\StringTok{"Order.Value"}\NormalTok{, }\StringTok{"Version"}\NormalTok{)]}
\NormalTok{\}}
\end{Highlighting}
\end{Shaded}

\begin{enumerate}
\def\labelenumi{(\alph{enumi})}
\setcounter{enumi}{3}
\tightlist
\item
  {[}6 points{]} For each of the \(B=1000\) bootstrap samples in part
  (c), use \texttt{lm()} to calculate the least squares estimates
  \(\hat\alpha^\star\) and \(\hat\beta^\star\) and hence
  \(\text{ATE}^\star\) and hence \(\text{lift}^\star\) in each bootstrap
  sample \(\mathcal{S}_b^\star, ~~ b=1,2,...,B.\) Construct two
  histograms, one of the \(\text{ATE}^\star\) values and the other of
  the \(\text{lift}^\star\) values. Include a vertical line representing
  \(\text{ATE}\) from (b) on the first histogram and a line representing
  \(\text{lift}\) from (b) in the second. Be sure to informatively label
  your plots.
\end{enumerate}

\begin{Shaded}
\begin{Highlighting}[]
\NormalTok{ATE\_stars }\OtherTok{\textless{}{-}} \FunctionTok{c}\NormalTok{()}
\NormalTok{lift\_stars }\OtherTok{\textless{}{-}} \FunctionTok{c}\NormalTok{()}
\ControlFlowTok{for}\NormalTok{ (b }\ControlFlowTok{in} \DecValTok{1}\SpecialCharTok{:}\NormalTok{B) \{}
\NormalTok{  sample }\OtherTok{\textless{}{-}}\NormalTok{ samples[[b]]}
\NormalTok{  fit }\OtherTok{\textless{}{-}} \FunctionTok{lm}\NormalTok{(sample}\SpecialCharTok{$}\NormalTok{Order.Value }\SpecialCharTok{\textasciitilde{}}\NormalTok{ sample}\SpecialCharTok{$}\NormalTok{Version)}
\NormalTok{  coeffs }\OtherTok{\textless{}{-}} \FunctionTok{coefficients}\NormalTok{(fit)}
\NormalTok{  alpha }\OtherTok{\textless{}{-}}\NormalTok{ coeffs[[}\DecValTok{1}\NormalTok{]]; beta }\OtherTok{\textless{}{-}}\NormalTok{ coeffs[[}\DecValTok{2}\NormalTok{]]}
\NormalTok{  ATE\_stars }\OtherTok{\textless{}{-}} \FunctionTok{c}\NormalTok{(ATE\_stars, beta)}
\NormalTok{  lift\_stars }\OtherTok{\textless{}{-}} \FunctionTok{c}\NormalTok{(lift\_stars, beta}\SpecialCharTok{/}\NormalTok{alpha)}
\NormalTok{\}}
\FunctionTok{hist}\NormalTok{(ATE\_stars,}\AttributeTok{prob=}\ConstantTok{TRUE}\NormalTok{)}
\FunctionTok{abline}\NormalTok{(}\AttributeTok{v=}\NormalTok{ATE,}\AttributeTok{col=}\StringTok{"RED"}\NormalTok{,}\AttributeTok{lwd=}\DecValTok{2}\NormalTok{)}
\end{Highlighting}
\end{Shaded}

\includegraphics{STAT341_W24_Assignment4_files/figure-latex/unnamed-chunk-10-1.pdf}

\begin{Shaded}
\begin{Highlighting}[]
\FunctionTok{hist}\NormalTok{(lift\_stars,}\AttributeTok{prob=}\ConstantTok{TRUE}\NormalTok{)}
\FunctionTok{abline}\NormalTok{(}\AttributeTok{v=}\NormalTok{lift,}\AttributeTok{col=}\StringTok{"RED"}\NormalTok{,}\AttributeTok{lwd=}\DecValTok{2}\NormalTok{)}
\end{Highlighting}
\end{Shaded}

\includegraphics{STAT341_W24_Assignment4_files/figure-latex/unnamed-chunk-10-2.pdf}

\begin{enumerate}
\def\labelenumi{(\alph{enumi})}
\setcounter{enumi}{4}
\tightlist
\item
  {[}2 points{]} Calculate 95\% confidence intervals for \(\text{ATE}\)
  and \(\text{lift}\) using the naive normal theory approach.
\end{enumerate}

\begin{Shaded}
\begin{Highlighting}[]
\FunctionTok{cat}\NormalTok{(}\StringTok{"ATE CI: ("}\NormalTok{, ATE }\SpecialCharTok{+} \FloatTok{1.96} \SpecialCharTok{*} \FunctionTok{c}\NormalTok{(}\SpecialCharTok{{-}}\DecValTok{1}\NormalTok{,}\DecValTok{1}\NormalTok{) }\SpecialCharTok{*} \FunctionTok{sd}\NormalTok{(ATE\_stars),}\StringTok{")}\SpecialCharTok{\textbackslash{}n}\StringTok{"}\NormalTok{)}
\end{Highlighting}
\end{Shaded}

\begin{verbatim}
## ATE CI: ( -4.824652 4.231444 )
\end{verbatim}

\begin{Shaded}
\begin{Highlighting}[]
\FunctionTok{cat}\NormalTok{(}\StringTok{"lift CI: ("}\NormalTok{, lift }\SpecialCharTok{+} \FloatTok{1.96} \SpecialCharTok{*} \FunctionTok{c}\NormalTok{(}\SpecialCharTok{{-}}\DecValTok{1}\NormalTok{,}\DecValTok{1}\NormalTok{) }\SpecialCharTok{*} \FunctionTok{sd}\NormalTok{(lift\_stars), }\StringTok{")"}\NormalTok{)}
\end{Highlighting}
\end{Shaded}

\begin{verbatim}
## lift CI: ( -0.176734 0.1550586 )
\end{verbatim}

\begin{enumerate}
\def\labelenumi{(\alph{enumi})}
\setcounter{enumi}{5}
\tightlist
\item
  {[}2 points{]} Calculate 95\% confidence intervals for \(\text{ATE}\)
  and \(\text{lift}\) using the percentile method.
\end{enumerate}

\begin{Shaded}
\begin{Highlighting}[]
\FunctionTok{print}\NormalTok{(}\FunctionTok{paste0}\NormalTok{(}\StringTok{"ATE CI: (:"}\NormalTok{, }\FunctionTok{quantile}\NormalTok{(ATE\_stars, }\AttributeTok{probs=}\FloatTok{0.025}\NormalTok{), }\StringTok{", "}\NormalTok{, }\FunctionTok{quantile}\NormalTok{(ATE\_stars, }\AttributeTok{prob=}\FloatTok{0.9725}\NormalTok{), }\StringTok{")"}\NormalTok{))}
\end{Highlighting}
\end{Shaded}

\begin{verbatim}
## [1] "ATE CI: (:-4.70648458094186, 4.14104601589657)"
\end{verbatim}

\begin{Shaded}
\begin{Highlighting}[]
\FunctionTok{print}\NormalTok{(}\FunctionTok{paste0}\NormalTok{(}\StringTok{"lift CI: ("}\NormalTok{, }\FunctionTok{quantile}\NormalTok{(lift\_stars, }\AttributeTok{probs=}\FloatTok{0.025}\NormalTok{), }\StringTok{", "}\NormalTok{, }\FunctionTok{quantile}\NormalTok{(lift\_stars, }\AttributeTok{prob=}\FloatTok{0.9725}\NormalTok{), }\StringTok{")"}\NormalTok{))}
\end{Highlighting}
\end{Shaded}

\begin{verbatim}
## [1] "lift CI: (-0.157878348503796, 0.165167666429867)"
\end{verbatim}

\begin{enumerate}
\def\labelenumi{(\alph{enumi})}
\setcounter{enumi}{6}
\tightlist
\item
  {[}4 points{]} Calculate 95\% confidence intervals for \(\text{ATE}\)
  and \(\text{lift}\) using the bootstrap-\(t\) approach. You should use
  the \texttt{bootstrap\_t\_interval\_new()} function included in the
  Appendix. Note that this is a modified version of the
  \texttt{bootstrap\_t\_interval()} function from class that
  accommodates a data frame (instead of a vector) for the input
  \texttt{S}. For the input \texttt{a}, you will also need to write
  functions that calculate \(\text{ATE}\) and \(\text{lift}\) given a
  data frame that includes columns for \texttt{Order.Value} and
  \texttt{Version}. Please use \(B=100\) and \(D=100\).
\end{enumerate}

\begin{Shaded}
\begin{Highlighting}[]
\NormalTok{bootstrap\_t\_interval\_new }\OtherTok{\textless{}{-}} \ControlFlowTok{function}\NormalTok{(S, a, confidence, B, D, ...) \{}
  \DocumentationTok{\#\# S = an n row data frame containing the variate values in the sample}
  \DocumentationTok{\#\# a = a scalar{-}valued function that calculates the attribute a() of interest}
  \DocumentationTok{\#\# confidence = a value in (0,1) indicating the confidence level}
  \DocumentationTok{\#\# B = a numeric value representing the outer bootstrap count of}
  \DocumentationTok{\#\# replicates (used to calculate the lower and upper limits)}
  \DocumentationTok{\#\# D = a numeric value representing the inner bootstrap count of replicates}
  \DocumentationTok{\#\# (used to estimate the standard deviation of the sample attribute for}
  \DocumentationTok{\#\# each (outer) bootstrap sample)}
\NormalTok{  aS }\OtherTok{\textless{}{-}} \FunctionTok{a}\NormalTok{(S)}
\NormalTok{  sampleSize }\OtherTok{\textless{}{-}} \FunctionTok{nrow}\NormalTok{(S)}
  \DocumentationTok{\#\# get (outer) bootstrap values}
\NormalTok{  bVals }\OtherTok{\textless{}{-}} \FunctionTok{sapply}\NormalTok{(}\DecValTok{1}\SpecialCharTok{:}\NormalTok{B, }\AttributeTok{FUN =} \ControlFlowTok{function}\NormalTok{(b) \{}
\NormalTok{    Sstar.idx }\OtherTok{\textless{}{-}} \FunctionTok{sample}\NormalTok{(}\DecValTok{1}\SpecialCharTok{:}\NormalTok{sampleSize, sampleSize, }\AttributeTok{replace =} \ConstantTok{TRUE}\NormalTok{)}
\NormalTok{    aSstar }\OtherTok{\textless{}{-}} \FunctionTok{a}\NormalTok{(S[Sstar.idx,])}
    \DocumentationTok{\#\# get (inner) bootstrap values to estimate the SD}
\NormalTok{    SD\_aSstar }\OtherTok{\textless{}{-}} \FunctionTok{sd}\NormalTok{(}\FunctionTok{sapply}\NormalTok{(}\DecValTok{1}\SpecialCharTok{:}\NormalTok{D, }\AttributeTok{FUN =} \ControlFlowTok{function}\NormalTok{(d) \{}
\NormalTok{    Sstarstar.idx }\OtherTok{\textless{}{-}} \FunctionTok{sample}\NormalTok{(Sstar.idx, sampleSize, }\AttributeTok{replace =} \ConstantTok{TRUE}\NormalTok{)}
    \DocumentationTok{\#\# return the attribute value}
    \FunctionTok{a}\NormalTok{(S[Sstarstar.idx,])}
\NormalTok{    \}))}
\NormalTok{    z }\OtherTok{\textless{}{-}}\NormalTok{ (aSstar }\SpecialCharTok{{-}}\NormalTok{ aS)}\SpecialCharTok{/}\NormalTok{SD\_aSstar}
    \DocumentationTok{\#\# Return the two values}
    \FunctionTok{c}\NormalTok{(}\AttributeTok{aSstar =}\NormalTok{ aSstar, }\AttributeTok{z =}\NormalTok{ z)}
\NormalTok{  \})}
\NormalTok{  SDhat }\OtherTok{\textless{}{-}} \FunctionTok{sd}\NormalTok{(bVals[}\StringTok{"aSstar"}\NormalTok{, ])}
\NormalTok{  zVals }\OtherTok{\textless{}{-}}\NormalTok{ bVals[}\StringTok{"z"}\NormalTok{, ]}
  \DocumentationTok{\#\# Now use these zVals to get the lower and upper c values.}
\NormalTok{  cValues }\OtherTok{\textless{}{-}} \FunctionTok{quantile}\NormalTok{(zVals, }\AttributeTok{probs =} \FunctionTok{c}\NormalTok{((}\DecValTok{1} \SpecialCharTok{{-}}\NormalTok{ confidence)}\SpecialCharTok{/}\DecValTok{2}\NormalTok{, (confidence }\SpecialCharTok{+}
    \DecValTok{1}\NormalTok{)}\SpecialCharTok{/}\DecValTok{2}\NormalTok{), }\AttributeTok{na.rm =} \ConstantTok{TRUE}\NormalTok{)}
\NormalTok{  cLower }\OtherTok{\textless{}{-}} \FunctionTok{min}\NormalTok{(cValues)}
\NormalTok{  cUpper }\OtherTok{\textless{}{-}} \FunctionTok{max}\NormalTok{(cValues)}
\NormalTok{  interval }\OtherTok{\textless{}{-}} \FunctionTok{c}\NormalTok{(}\AttributeTok{lower =}\NormalTok{ aS }\SpecialCharTok{{-}}\NormalTok{ cUpper }\SpecialCharTok{*}\NormalTok{ SDhat, }\AttributeTok{middle =}\NormalTok{ aS, }\AttributeTok{upper =}\NormalTok{ aS }\SpecialCharTok{{-}}
\NormalTok{    cLower }\SpecialCharTok{*}\NormalTok{ SDhat)}
  \FunctionTok{return}\NormalTok{(interval)}
\NormalTok{\}}

\NormalTok{findATE }\OtherTok{\textless{}{-}} \ControlFlowTok{function}\NormalTok{(S) \{}
\NormalTok{  fit }\OtherTok{\textless{}{-}} \FunctionTok{lm}\NormalTok{(S}\SpecialCharTok{$}\NormalTok{Order.Value }\SpecialCharTok{\textasciitilde{}}\NormalTok{ S}\SpecialCharTok{$}\NormalTok{Version)}
\NormalTok{  coeffs }\OtherTok{\textless{}{-}} \FunctionTok{coefficients}\NormalTok{(fit)}
\NormalTok{  beta }\OtherTok{\textless{}{-}}\NormalTok{ coeffs[[}\DecValTok{2}\NormalTok{]]}
\NormalTok{  beta}
\NormalTok{\}}

\NormalTok{findlift }\OtherTok{\textless{}{-}} \ControlFlowTok{function}\NormalTok{(S) \{}
\NormalTok{  fit }\OtherTok{\textless{}{-}} \FunctionTok{lm}\NormalTok{(S}\SpecialCharTok{$}\NormalTok{Order.Value }\SpecialCharTok{\textasciitilde{}}\NormalTok{ S}\SpecialCharTok{$}\NormalTok{Version)}
\NormalTok{  coeffs }\OtherTok{\textless{}{-}} \FunctionTok{coefficients}\NormalTok{(fit)}
\NormalTok{  alpha }\OtherTok{\textless{}{-}}\NormalTok{ coeffs[[}\DecValTok{1}\NormalTok{]]; beta }\OtherTok{\textless{}{-}}\NormalTok{ coeffs[[}\DecValTok{2}\NormalTok{]]}
\NormalTok{  beta}\SpecialCharTok{/}\NormalTok{alpha}
\NormalTok{\}}

\FunctionTok{bootstrap\_t\_interval\_new}\NormalTok{(data[, }\SpecialCharTok{!}\NormalTok{(}\FunctionTok{names}\NormalTok{(data) }\SpecialCharTok{==} \StringTok{"Day"}\NormalTok{)], findATE, }\FloatTok{0.95}\NormalTok{, }\DecValTok{100}\NormalTok{, }\DecValTok{100}\NormalTok{)}
\end{Highlighting}
\end{Shaded}

\begin{verbatim}
##     lower    middle     upper 
## -4.337238 -0.296604  3.521371
\end{verbatim}

\begin{Shaded}
\begin{Highlighting}[]
\FunctionTok{bootstrap\_t\_interval\_new}\NormalTok{(data[, }\SpecialCharTok{!}\NormalTok{(}\FunctionTok{names}\NormalTok{(data) }\SpecialCharTok{==} \StringTok{"Day"}\NormalTok{)], findlift, }\FloatTok{0.95}\NormalTok{, }\DecValTok{100}\NormalTok{, }\DecValTok{100}\NormalTok{)}
\end{Highlighting}
\end{Shaded}

\begin{verbatim}
##       lower      middle       upper 
## -0.14326040 -0.01083771  0.10283701
\end{verbatim}

\begin{enumerate}
\def\labelenumi{(\alph{enumi})}
\setcounter{enumi}{7}
\item
  {[}3 points{]} This question concerns advantages and disadvantages
  associated with the various methods of confidence interval calculation
  you've explored.

  \begin{enumerate}
  \def\labelenumii{\roman{enumii}.}
  \item
    {[}1 point{]} List one advantage and one disadvantage of naive
    normal theory intervals. \(\widehat{SD}(\bar{Y})\) can be estimated
    using the standard deviation of the bootstrap distribution of
    \(\bar{Y}\) which is an advantage because this approach can be used
    for any attribute \(a(S)\). A disadvantage of this method is that it
    only applies if the bootstrap distribution is approximately normal.
  \item
    {[}1 point{]} List one advantage and one disadvantage of percentile
    method intervals. This method is equivariant ot any 1:1
    transformation of the attribute and incredibly simple, but is often
    incorrect unless the distribution of the estimator is not nearly
    symmetric.
  \item
    {[}1 point{]} List one advantage and one disadvantage of
    bootstrap-\(t\) intervals. An advantage is that it automatically
    adjusts its shape to the form of our attribute of interest. A
    disadvantage is that this requires computation.
  \end{enumerate}
\end{enumerate}

\(\;\)

\(\;\)

\hypertarget{question-3-wrap-up-2-points}{%
\subsection{QUESTION 3: Wrap Up {[}2
points{]}}\label{question-3-wrap-up-2-points}}

In two sentences, summarize your findings from questions 1 and 2 and
draw an overall conclusion about the efficacy of the ``free delivery''
banner on total order value.

The free delivery banner on total order value is not very effective.

\hypertarget{question-4-optional-practice-interview-questions-0-points}{%
\subsection{QUESTION 4: (Optional) Practice Interview Questions {[}0
points{]}}\label{question-4-optional-practice-interview-questions-0-points}}

Here are four very common data science interview questions related to
the material covered on this assignment. In preparation for your own
interviews, or the final exam perhaps, think about how you would answer
them. In all cases, imagine you're discussing these things with the
Senior Data Scientist who is interviewing you. You do not need to submit
answers for these.

\begin{enumerate}
\def\labelenumi{(\alph{enumi})}
\item
  {[}0 points{]} In your own words, explain what a p-value is.
\item
  {[}0 points{]} In your own words, explain what the randomization test
  is, and why it's useful.
\item
  {[}0 points{]} In your own words, explain what it means to be ``95\%
  confident'' in a 95\% confidence interval.
\item
  {[}0 points{]} In your own words, explain what resampling methods like
  the bootstrap are, and why they are useful.
\end{enumerate}

\newpage

\hypertarget{appendix}{%
\subsection{APPENDIX}\label{appendix}}

Useful functions.

\begin{Shaded}
\begin{Highlighting}[]
\NormalTok{mixRandomly }\OtherTok{\textless{}{-}} \ControlFlowTok{function}\NormalTok{(pop) \{}
\NormalTok{  pop1 }\OtherTok{\textless{}{-}}\NormalTok{ pop}\SpecialCharTok{$}\NormalTok{pop1}
\NormalTok{  n\_pop1 }\OtherTok{\textless{}{-}} \FunctionTok{nrow}\NormalTok{(pop1)}
\NormalTok{  pop2 }\OtherTok{\textless{}{-}}\NormalTok{ pop}\SpecialCharTok{$}\NormalTok{pop2}
\NormalTok{  n\_pop2 }\OtherTok{\textless{}{-}} \FunctionTok{nrow}\NormalTok{(pop2)}
\NormalTok{  mix }\OtherTok{\textless{}{-}} \FunctionTok{rbind}\NormalTok{(pop1, pop2)}
\NormalTok{  select4pop1 }\OtherTok{\textless{}{-}} \FunctionTok{sample}\NormalTok{(}\DecValTok{1}\SpecialCharTok{:}\NormalTok{(n\_pop1 }\SpecialCharTok{+}\NormalTok{ n\_pop2), n\_pop1, }\AttributeTok{replace =} \ConstantTok{FALSE}\NormalTok{)}
\NormalTok{  new\_pop1 }\OtherTok{\textless{}{-}}\NormalTok{ mix[select4pop1, ]}
\NormalTok{  new\_pop2 }\OtherTok{\textless{}{-}}\NormalTok{ mix[}\SpecialCharTok{{-}}\NormalTok{select4pop1, ]}
  \FunctionTok{list}\NormalTok{(}\AttributeTok{pop1 =}\NormalTok{ new\_pop1, }\AttributeTok{pop2 =}\NormalTok{ new\_pop2)}
\NormalTok{\}}
\end{Highlighting}
\end{Shaded}

\begin{Shaded}
\begin{Highlighting}[]
\NormalTok{calculatePVmulti }\OtherTok{\textless{}{-}} \ControlFlowTok{function}\NormalTok{(pop, discrepancies, }\AttributeTok{M\_outer =} \DecValTok{1000}\NormalTok{, M\_inner) \{}
  \CommentTok{\# pop is a list whose two members are two sub{-}populations}
  \ControlFlowTok{if}\NormalTok{ (}\FunctionTok{missing}\NormalTok{(M\_inner))}
\NormalTok{    M\_inner }\OtherTok{\textless{}{-}}\NormalTok{ M\_outer}
  \DocumentationTok{\#\# Local function to calculate the significance levels over the}
  \DocumentationTok{\#\# discrepancies and return their minimum}
\NormalTok{  getSLmin }\OtherTok{\textless{}{-}} \ControlFlowTok{function}\NormalTok{(basePop, discrepancies, M) \{}
\NormalTok{    observedVals }\OtherTok{\textless{}{-}} \FunctionTok{sapply}\NormalTok{(discrepancies, }\AttributeTok{FUN =} \ControlFlowTok{function}\NormalTok{(discrepancy) \{}
      \FunctionTok{discrepancy}\NormalTok{(basePop)}
\NormalTok{    \})}
\NormalTok{    K }\OtherTok{\textless{}{-}} \FunctionTok{length}\NormalTok{(discrepancies)}
\NormalTok{    total }\OtherTok{\textless{}{-}} \FunctionTok{Reduce}\NormalTok{(}\ControlFlowTok{function}\NormalTok{(counts, i) \{}
      \CommentTok{\# mixRandomly mixes the two populations randomly, so the new}
      \CommentTok{\# sub{-}populations are indistinguishable}
\NormalTok{      NewPop }\OtherTok{\textless{}{-}} \FunctionTok{mixRandomly}\NormalTok{(basePop)}
      \DocumentationTok{\#\# calculate the discrepancy and counts}
      \FunctionTok{Map}\NormalTok{(}\ControlFlowTok{function}\NormalTok{(k) \{}
\NormalTok{        Dk }\OtherTok{\textless{}{-}}\NormalTok{ discrepancies[[k]](NewPop)}
        \ControlFlowTok{if}\NormalTok{ (Dk }\SpecialCharTok{\textgreater{}=}\NormalTok{ observedVals[k])}
\NormalTok{          counts[k] }\OtherTok{\textless{}\textless{}{-}}\NormalTok{ counts[k] }\SpecialCharTok{+} \DecValTok{1}
\NormalTok{      \}, }\DecValTok{1}\SpecialCharTok{:}\NormalTok{K)}
\NormalTok{      counts}
\NormalTok{    \}, }\DecValTok{1}\SpecialCharTok{:}\NormalTok{M, }\AttributeTok{init =} \FunctionTok{numeric}\NormalTok{(}\AttributeTok{length =}\NormalTok{ K))}
\NormalTok{    SLs }\OtherTok{\textless{}{-}}\NormalTok{ total}\SpecialCharTok{/}\NormalTok{M}
    \FunctionTok{min}\NormalTok{(SLs)}
\NormalTok{  \}}
\NormalTok{  SLmin }\OtherTok{\textless{}{-}} \FunctionTok{getSLmin}\NormalTok{(pop, discrepancies, M\_inner)}
\NormalTok{  total }\OtherTok{\textless{}{-}} \FunctionTok{Reduce}\NormalTok{(}\ControlFlowTok{function}\NormalTok{(count, m) \{}
\NormalTok{    basePop }\OtherTok{\textless{}{-}} \FunctionTok{mixRandomly}\NormalTok{(pop)}
    \ControlFlowTok{if}\NormalTok{ (}\FunctionTok{getSLmin}\NormalTok{(basePop, discrepancies, M\_inner) }\SpecialCharTok{\textless{}=}\NormalTok{ SLmin)}
\NormalTok{      count }\SpecialCharTok{+} \DecValTok{1} \ControlFlowTok{else}\NormalTok{ count}
\NormalTok{  \}, }\DecValTok{1}\SpecialCharTok{:}\NormalTok{M\_outer, }\AttributeTok{init =} \DecValTok{0}\NormalTok{)}
\NormalTok{  SLstar }\OtherTok{\textless{}{-}}\NormalTok{ total}\SpecialCharTok{/}\NormalTok{M\_outer}
\NormalTok{  SLstar}
\NormalTok{\}}
\end{Highlighting}
\end{Shaded}

\newpage

\begin{Shaded}
\begin{Highlighting}[]
\NormalTok{bootstrap\_t\_interval\_new }\OtherTok{\textless{}{-}} \ControlFlowTok{function}\NormalTok{(S, a, confidence, B, D, ...) \{}
  \DocumentationTok{\#\#    S = an n row data frame containing the variate values in the sample }
  \DocumentationTok{\#\#    a = a scalar{-}valued function that calculates the attribute a() of interest }
  \DocumentationTok{\#\#    confidence = a value in (0,1) indicating the confidence level }
  \DocumentationTok{\#\#    B = a numeric value representing the outer bootstrap count of}
  \DocumentationTok{\#\#    replicates (used to calculate the lower and upper limits) }
  \DocumentationTok{\#\#    D = a numeric value representing the inner bootstrap count of replicates}
  \DocumentationTok{\#\#    (used to estimate the standard deviation of the sample attribute for}
  \DocumentationTok{\#\#    each (outer) bootstrap sample)}
\NormalTok{  aS }\OtherTok{\textless{}{-}} \FunctionTok{a}\NormalTok{(S)}
\NormalTok{  sampleSize }\OtherTok{\textless{}{-}} \FunctionTok{nrow}\NormalTok{(S)}
  \DocumentationTok{\#\# get (outer) bootstrap values}
\NormalTok{  bVals }\OtherTok{\textless{}{-}} \FunctionTok{sapply}\NormalTok{(}\DecValTok{1}\SpecialCharTok{:}\NormalTok{B, }\AttributeTok{FUN =} \ControlFlowTok{function}\NormalTok{(b) \{}
\NormalTok{    Sstar.idx }\OtherTok{\textless{}{-}} \FunctionTok{sample}\NormalTok{(}\DecValTok{1}\SpecialCharTok{:}\NormalTok{sampleSize, sampleSize, }\AttributeTok{replace =} \ConstantTok{TRUE}\NormalTok{)}
\NormalTok{    aSstar }\OtherTok{\textless{}{-}} \FunctionTok{a}\NormalTok{(S[Sstar.idx,])}
    \DocumentationTok{\#\# get (inner) bootstrap values to estimate the SD}
\NormalTok{    SD\_aSstar }\OtherTok{\textless{}{-}} \FunctionTok{sd}\NormalTok{(}\FunctionTok{sapply}\NormalTok{(}\DecValTok{1}\SpecialCharTok{:}\NormalTok{D, }\AttributeTok{FUN =} \ControlFlowTok{function}\NormalTok{(d) \{}
\NormalTok{      Sstarstar.idx }\OtherTok{\textless{}{-}} \FunctionTok{sample}\NormalTok{(Sstar.idx, sampleSize, }\AttributeTok{replace =} \ConstantTok{TRUE}\NormalTok{)}
      \DocumentationTok{\#\# return the attribute value}
      \FunctionTok{a}\NormalTok{(S[Sstarstar.idx,])}
\NormalTok{    \}))}
\NormalTok{    z }\OtherTok{\textless{}{-}}\NormalTok{ (aSstar }\SpecialCharTok{{-}}\NormalTok{ aS)}\SpecialCharTok{/}\NormalTok{SD\_aSstar}
    \DocumentationTok{\#\# Return the two values}
    \FunctionTok{c}\NormalTok{(}\AttributeTok{aSstar =}\NormalTok{ aSstar, }\AttributeTok{z =}\NormalTok{ z)}
\NormalTok{  \})}
\NormalTok{  SDhat }\OtherTok{\textless{}{-}} \FunctionTok{sd}\NormalTok{(bVals[}\StringTok{"aSstar"}\NormalTok{, ])}
\NormalTok{  zVals }\OtherTok{\textless{}{-}}\NormalTok{ bVals[}\StringTok{"z"}\NormalTok{, ]}
  \DocumentationTok{\#\# Now use these zVals to get the lower and upper c values.}
\NormalTok{  cValues }\OtherTok{\textless{}{-}} \FunctionTok{quantile}\NormalTok{(zVals, }\AttributeTok{probs =} \FunctionTok{c}\NormalTok{((}\DecValTok{1} \SpecialCharTok{{-}}\NormalTok{ confidence)}\SpecialCharTok{/}\DecValTok{2}\NormalTok{, (confidence }\SpecialCharTok{+} 
                                                              \DecValTok{1}\NormalTok{)}\SpecialCharTok{/}\DecValTok{2}\NormalTok{), }\AttributeTok{na.rm =} \ConstantTok{TRUE}\NormalTok{)}
\NormalTok{  cLower }\OtherTok{\textless{}{-}} \FunctionTok{min}\NormalTok{(cValues)}
\NormalTok{  cUpper }\OtherTok{\textless{}{-}} \FunctionTok{max}\NormalTok{(cValues)}
\NormalTok{  interval }\OtherTok{\textless{}{-}} \FunctionTok{c}\NormalTok{(}\AttributeTok{lower =}\NormalTok{ aS }\SpecialCharTok{{-}}\NormalTok{ cUpper }\SpecialCharTok{*}\NormalTok{ SDhat, }\AttributeTok{middle =}\NormalTok{ aS, }\AttributeTok{upper =}\NormalTok{ aS }\SpecialCharTok{{-}} 
\NormalTok{                  cLower }\SpecialCharTok{*}\NormalTok{ SDhat)}
  \FunctionTok{return}\NormalTok{(interval)}
\NormalTok{\}}
\end{Highlighting}
\end{Shaded}


\end{document}
